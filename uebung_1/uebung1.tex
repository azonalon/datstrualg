\documentclass[11pt]{article}
\usepackage[utf8]{inputenc}
\usepackage[english, ngerman]{babel}
\usepackage{amsmath,amsthm,verbatim,amssymb,amsfonts,amscd}
\usepackage{enumerate}
\usepackage{listings}
\usepackage{courier}
\lstset{
  numbers=left,
  language=C,
  basicstyle=\footnotesize\ttfamily,
  breaklines=true,
  morekeywords={function}
}
\newcommand{\abs}[1]{\left| #1 \right| }


\author{Felix Schrader, Jens Duffert (2.Mastersemester,Mtr.: 2843110), Eduard Sauter}
\title{Datenstrukturen und Algorithmen: Haus\"ubung 1}
\begin{document}
\maketitle

\subsection*{Aufgabe 1.1}
\begin{enumerate}[a)]

\item Das folgende Programm berechnet den Binoialkoeffizienten rekursiv:

\begin{lstlisting}
// Funktion definieren:
function bina(n, k) {
  // Ergebnisse fuer k=0 und n=k eins setzen, weil bin(n,0)=bin(n,n)=1.
  if(k == 0) return 1;
  if(n == k) return 1;
  // Ansonsten die Rekursionsformel benutzen:
  return bina(n-1, k-1) + bina(n-1, k);
}
\end{lstlisting}

\item Da die einzigen Startwerte immer eins sind und immer addiert werden um den Binomialkoeffizienten zu berechnen, dient das Ergebnis selber als Counter. Wenn das Programm den Koeffizienten berechnet, so sieht die Formel effektiv so aus:\[ 1+1+1+\dots =\binom{n}{k} \]
Man kann leicht erkennen, das daf\"ur also
\begin{equation}
N_{Add}=\binom{n}{k}-1
\end{equation}
Additionen n\"otig sind.

\item Das folgende Programm verwendet bei der Berechnung Memoisation:

\begin{lstlisting}
// Funktion definieren:
function binc(n, k) {
    // (n kreuz k)-Matrix erstellen (alle Elemente sind undefined). Darin sollen bekannte Ergebnisse gespeichert werden.
    B = new Array(n);
    for(i = 0; i <= n; i++) {
        B[i] = new Array(k);
    }
    // Eintraege mit k=0 und n=k eins setzen:
    for(nIndex = 0; nIndex <= n; nIndex++) {
        for(kIndex = 0; kIndex <= k; kIndex++) {
            if(kIndex == 0) B[nIndex][kIndex]=1;
            if(nIndex == kIndex) B[nIndex][kIndex]=1;
        }
    }
    // Funktion, die den Binomialkoeffizienten mit Memoisation berechnet, aufrufen:
    return memobin(n, k, B)
}

// memobin definieren:
function memobin(n, k, B) {
    // Wenn B[n][k] schon bekannt ist (also typeof(B[n][k]) nicht mehr "undefined" ist, soll dieser Wert ausgegeben und die Funktion verlassen werden:
    if(typeof(B[n][k]) == "number") return B[n][k];
    // Ansonsten den Wert mit der Rekursionsformel bestimmen und als b speichern:
    b = memobin(n-1, k-1, B) + memobin(n-1, k, B);
    // Diesen Wert fuer evtl. spaetere Berechnungen in die Matrix eintragen und als Ergebnis ausgeben:
    B[n][k] = b;
    return b;
}
\end{lstlisting}

\end{enumerate}

\subsection*{Aufgabe 1.2}
\begin{enumerate}[a)]
  \item
    Das folgende Programm berechnet die Quadratwurzel einer positiven
    Zahl $x$ nach dem Divide-and-Conquer Paradigma.


    \begin{lstlisting}
function SqrtBisect(x, eps){
    return SqrtBisectRecurse(x, 0, x + 1, eps * eps);
}

function SqrtBisectRecurse(x, a, b, eps){
    guess = (b+a)/2;
    square = guess*guess;
    if (abs(x - square) < eps)
        return guess;
    else if (square < x)
        return SqrtBisectRecurse(x, guess, b, eps);
    else
        return SqrtBisectRecurse(x, a, guess,  eps);
}
  \end{lstlisting}

  Um die Korrektheit zu zeigen, wird vorrausgesetzt ,dass
  \begin{equation}
    a < b \iff a^2 < b^2 \quad\forall\, a, b \ge 0
  \end{equation}
  gilt.

  \texttt{SqrtBisect} sucht $\sqrt{x}$ im Intervall $[0,x+1]$. Die Quadratwurzel
  muss in diesem Bereich liegen, da $a \times a > a \quad\forall\, a \ge 1 $ und
  $a \times a \le 1 \quad\forall\, 0 \le a \le 1$.
  Zu beginn eines Funktionsaufrufes sei also sichergestellt, dass $\sqrt{x} \in
  [a,b]$.  Aufgrund der oberen Gleichung und Zeilen 10-13 bleibt dies
  Gew\"ahrleistet. Die Abbruchbedingung muss somit erreicht werden, da das
  Suchintervall nach jedem Schritt halbiert wird. An dieser Stelle ist
  % \texttt{guess}$^2$ im bereich $\varepsilon^2$ um $x$. \\
  \begin{equation}
  | x - \texttt{guess}^2 | < \varepsilon^2
  \end{equation}
  Es bleibt noch zu zeigen, dass daraus folgt $\left| \sqrt{x} - \texttt{guess}
  < \varepsilon \right|$.

  Dies ergibt sich aus \[ \left| x-a \right|^2 \le \left| x^2 - a^2 \right|  \]
  \emph{Beweis}
  \begin{enumerate}
    \item Fall $x \ge a$
      \[ \abs{x-a}^2 = x^2 + a^2 - 2 ax \le x^2 + a^2 - 2 a^2 = x^2 -a^2 =
      \abs{x^2-a^2}  \]
    \item Fall $x < a$
      \[ \abs{x-a}^2 = x^2 + a^2 - 2 ax \le x^2 + a^2 - 2 x^2 = a^2 -x^2 =
      \abs{x^2-a^2}  \]
  \end{enumerate}
  Damit ist $\abs{ \texttt{SqrtBisect}(x)- \sqrt{x} } < \varepsilon$.

  \item Im schlimmsten Fall erreicht $b-a$ den Wert von $\varepsilon^2$.
    Dies geschieht, falls gleich zu begin $a=\sqrt{x}=0$. Da nach jedem Schritt
    $b-a \to (b-a)/2$ ist nach dem $n$-ten Schritt f\"ur den Startwert
    von $b-a=x+1$ der Wert $\frac{x+1}{2^n}$. Man erh\"alt als L\"osung:
    \[ \frac{x+1}{2^n} = \varepsilon \implies n= \lg{\frac{x+1}{\varepsilon}} \]


\end{enumerate}

% 42 nice !
\subsection*{Aufgabe 1.3}
Nimmt man z.B. die Liste $\lbrace 1, 19, 21, 22\rbrace$, so arbeitet die
Coin-Changing-Methode mit dem Greedy-Algorithmus f\"ur die Zerlegung der Zahl
42 nicht optimal. W\"urde man diesen Algorithmus verwenden, w\"urde zun\"achst
die 22 aus der Liste ausgew\"ahlt. Der Rest sind 20. Also w\"urde im n\"achsten
Schritt die 19 ausgew\"ahlt. Der Rest 1 wird durch die 1 aufgef\"ullt. Man
braucht also 3 Elemente der Liste.

Optimal w\"are allerdings die Zerlegung $42 = 21 + 21$, da dort nur 2 Elemente
ben\"otigt werden. Diese Zerlegung findet der Greedy-Algorithmus aber nicht.

\end{document}
