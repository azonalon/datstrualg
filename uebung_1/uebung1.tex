\documentclass[11pt]{article}
\usepackage[utf8]{inputenc}
\usepackage[english, ngerman]{babel}
\usepackage{amsmath,amsthm,verbatim,amssymb,amsfonts,amscd}
\usepackage{enumerate}
\usepackage{listings}
\usepackage{courier}
\lstset{
  numbers=left,
  language=C,
  basicstyle=\footnotesize\ttfamily,
  breaklines=true,
  morekeywords={function}
}
\newcommand{\abs}[1]{\left| #1 \right| }


\author{Felix Schrader, Jens Duffert, Eduard Sauter}
\title{Datenstrukturen und Algorithmen: Haus\"ubung 1}
\begin{document}
\maketitle

\subsection*{Aufgabe 1.2}
\begin{enumerate}[a)]
  \item
    Das folgende Programm berechnet die Quadratwurzel einer positiven
    Zahl $x$ nach dem Divide-and-Conquer Paradigma.


    \begin{lstlisting}
function SqrtBisect(x, eps){
    return SqrtBisectRecurse(x, 0, x + 1, eps * eps);
}

function SqrtBisectRecurse(x, a, b, eps){
    guess = (b+a)/2;
    square = guess*guess;
    if (abs(x - square) < eps)
        return guess;
    else if (square < x)
        return SqrtBisectRecurse(x, guess, b, eps);
    else
        return SqrtBisectRecurse(x, a, guess,  eps);
}
  \end{lstlisting}

  Um die Korrektheit zu zeigen, wird vorrausgesetzt ,dass
  \begin{equation}
    a < b \iff a^2 < b^2 \quad\forall\, a, b \ge 0
  \end{equation}
  gilt.

  \texttt{SqrtBisect} sucht $\sqrt{x}$ im Intervall $[0,x+1]$. Die Quadratwurzel
  muss in diesem Bereich liegen, da $a * a > a \quad\forall\, a \ge 1 $ und
  $a * a \le 1 \quad\forall\, 0 \le a \le 1$.
  Zu beginn eines Funktionsaufrufes sei also sichergestellt, dass $\sqrt{x} \in
  [a,b]$.  Aufgrund der oberen Gleichung und Zeilen 10-13 bleibt dies
  Gew\"ahrleistet. Die Abbruchbedingung muss somit erreicht werden, da das
  Suchintervall nach jedem Schritt halbiert wird. An dieser Stelle ist
  % \texttt{guess}$^2$ im bereich $\varepsilon^2$ um $x$. \\
  \begin{equation}
  | x - \texttt{guess}^2 | < \varepsilon^2
  \end{equation}
  Es bleibt noch zu zeigen, dass daraus folgt $\left| \sqrt{x} - \texttt{guess}
  < \varepsilon \right|$.

  Dies ergibt sich aus \[ \left| x-a \right|^2 \le \left| x^2 - a^2 \right|  \]
  \emph{Beweis}
  \begin{enumerate}
    \item Fall $x \ge a$
      \[ \abs{x-a}^2 = x^2 + a^2 - 2 ax \le x^2 + a^2 - 2 a^2 = x^2 -a^2 =
      \abs{x^2-a^2}  \]
    \item Fall $x < a$
      \[ \abs{x-a}^2 = x^2 + a^2 - 2 ax \le x^2 + a^2 - 2 x^2 = a^2 -x^2 =
      \abs{x^2-a^2}  \]
  \end{enumerate}
  Damit ist $\abs{ \texttt{SqrtBisect}(x)- \sqrt{x} } < \varepsilon$.

  \item Im schlimmsten Fall erreicht $b-a$ den Wert von $\varepsilon^2$.
    Dies geschieht, falls gleich zu begin $a=\sqrt{x}=0$. Da nach jedem Schritt
    $b-a \to (b-a)/2$ ist nach dem $n$-ten Schritt f\"ur den Startwert
    von $b-a=x+1$ der Wert $\frac{x+1}{2^n}$. Man erh\"alt als L\"osung:
    \[ \frac{x+1}{2^n} = \varepsilon \implies n= \lg{\frac{x+1}{\varepsilon}} \]


\end{enumerate}



\end{document}
