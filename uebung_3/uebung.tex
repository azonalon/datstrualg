\documentclass[11pt]{article}
\usepackage[utf8]{inputenc}
\usepackage[english, ngerman]{babel}
\usepackage{amsmath,amsthm,verbatim,amssymb,amsfonts,amscd}
\usepackage{enumerate}
\usepackage{listings}
\usepackage{courier}
\usepackage[margin=1in]{geometry}
\lstset{
  numbers=left,
  language=C,
  basicstyle=\footnotesize\ttfamily,
  breaklines=true,
  morekeywords={function, NIL}
}
\newcommand{\abs}[1]{\left| #1 \right| }
\setlength{\parindent}{0pt} 

\author{
  Felix Schrader, 3053850 \\ 
  Jens Duffert, 2843110 \\
  Eduard Sauter, 3053470
}
\title{Datenstrukturen und Algorithmen: Haus\"ubung 3}
\begin{document}
\maketitle
\subsection*{Aufgabe 1}
\begin{enumerate}[a)]
\item
  Der folgende Algorithmus legt zunächst ein neues Array
  \texttt{permuted$\_$data} an, das mit dem Ergebnis der Permutation gefüllt
  werden soll. Dazu wird mit einer \texttt{for}-Schleife das ganze Array
  durchgegangen. Die Stelle \texttt{index} in \texttt{permuted$\_$data} wird
  dann mit \texttt{data$\lbrack$permutation$\lbrack$index$\rbrack\rbrack$}
  gefüllt, da dies der Wert ist, der nach dem permutieren an der Stelle
  \texttt{index} stehen soll.
  \begin{lstlisting}
    function permute(data, permutation) {
        permuted_data = new Array(n);
        for(index = 0, index < n, index++) {
            permuted_data[index] = data[permutation[index]];
        }
        return permuted_data;
    }
  \end{lstlisting}
  In dem Algorithmus gibt es zunächst eine Zuweisung in der Zeile:
  \begin{lstlisting}
    permuted_data = new Array(n);
  \end{lstlisting}
  Dann wird eine \texttt{for}-Schleife durchlaufen:
  \begin{lstlisting}
    for(index = 0, index < n, index++) {
        permuted_data[index] = data[permutation[index]];
    }
  \end{lstlisting}
  Der Schleifenkopf weist dann mit der Initialisierung einmal einen Wert zu,
  vergleicht dann $n$-mal, ob \texttt{index} kleiner als $n$ ist und erhöht
  \texttt{index} dann $n$-mal, was je einer Addition und einer Zuweisung
  entspricht. Dann wird der entsprechenden Stelle in \texttt{permuted$\_$data}
  ein Wert zugewiesen. Auch das passiert $n$-mal. Insgesamt erhält man also
  eine Laufzeit:
  \begin{align*}
    f (n) &= 1 + n + n + 2 \cdot n + n
    \\ &= 5 \cdot n + 1 \in \Theta (n)
  \end{align*}
  Allerdings benötigt man auch linear viel Speicher, da man ein neues Array der
  Länge $n$ erstellt.
\item 
\end{enumerate}
\subsection*{Aufgabe 2}
\begin{enumerate}[a)]
  \item  $ $
\begin{center}
\begin{lstlisting}
function insertList(L1, L2, pos) {
  L1_position = pos;
  L2_position = L2.first();

  while(L2_position != NIL) {
    L1.insert(L1_position, L2.retrieve(L2_position));
    L1_position = L1_position.next();
    L2_position = L2_position.next();
  }
}
\end{lstlisting}
\end{center}
    
  \item Laufzeit Untersuchungen:
    \begin{enumerate}[i)]
      \item
        Nach Vorlesung sind die Operationen \texttt{insert}, \texttt{next}, 
        \texttt{retrieve} und \texttt{first} implementiert durch die verkettete
        Liste in $\mathcal{O}(1)$. Es sei $n=\texttt{L2.length()}$. 
        Die Schleife beginnend bei Zeile 5 wird $n$ mal durchlaufen. Jede der 
        Operationen im Schleifenkopf sowie der Vergleich zu beginn der Schleife
        sind in $\mathcal{O}(1)$. Folglich ist die Laufzeit von \texttt{insertList}
        in $\mathcal{O}(n)$.
      \item W\"urde man \texttt{L2} vor \texttt{pos} einf\"ugen wollen, so
        m\"usste man vorher die Position vor \texttt{pos} finden.
\begin{lstlisting}
function previous(L, pos) {
    last_pos = NIL; current_pos = L.first(); while(current_pos != pos && current_pos != NIL) {
        last_pos = current_pos;
        current_pos = pos.next();
    }
    return last_pos
}
\end{lstlisting}
        Es sei $m=\texttt{L1.Length}$ 
        Die Worst-Case Laufzeit von \texttt{previous} ist in $\mathcal{O}(m)$,
        da in diesem Fall $m$ Elemente \"uberpr\"uft werden m\"ussen. 

\begin{lstlisting}
  function insertListPrevious(L1, L2, pos) {
      L1_position = previous(L1, pos);
      L2_position = L2.first();
  
      while(L2_position != NIL) {
          L1.insert(L1_position, L2.retrieve(L2_position));
          L1_position = L1_position.next();
          L2_position = L2_position.next();
      }
  }
\end{lstlisting}
        
        Analog zu \texttt{insertList} ergibt sich die Worst-Case Laufzeit von
        \texttt{InsertListPrevious} zu $\mathcal{O}(m + n)$.

  \end{enumerate}
\end{enumerate}
\newpage
\subsection*{Aufgabe 3}
\begin{enumerate}[a)]
  \item 
    Die Methoden sehen wie folgt aus (Quelle: http://en.cppreference.com/w/cpp/container/vector). 
    Es sein \texttt{v} ein \texttt{std::vector}: 

    \begin{centering}
      \begin{tabular}{r c l c}
        \texttt{L.first}     & $\leftrightarrow$ & \texttt{0}       &   $\mathcal{O}(1)$\\
        \texttt{L.last}      & $\leftrightarrow$ & \texttt{v.size-1} &   $\mathcal{O}(1)$\\
        \texttt{L.length}    & $\leftrightarrow$ & \texttt{v.size  }   &   $\mathcal{O}(1)$\\
        \texttt{L.retrieve}  & $\leftrightarrow$ & \texttt{v.at    }   &   $\mathcal{O}(1)$\\
        \texttt{L.delete}    & $\leftrightarrow$ & \texttt{v.erase }  &   $\mathcal{O}(n)$\\
        \texttt{L.insert(p)}    & $\leftrightarrow$ & \texttt{v.insert(p+1, x)} &   $\mathcal{O}(n)$
      \end{tabular}
    \end{centering}

    Der \texttt{std::vector} arbeitet \"ahnlich wie die ADT Liste,
    die positionen sind jedoch integer indizes. F\"ur die letztendliche
    implementierung sind noch grenz\"uberpr\"ufungen f\"ur den Zugriff
    zu beachten, hier sind nur die Funktionen gelistet, mit welchen eine 
    ADT Liste schnell durch einen Vektor implementiert werden kann.
    

  \item

    Die \texttt{std::deque} ist prinzipiell ein Listen\"ahnlicher Typ,
    welcher es erm\"oglicht, schnell Elemente am beginn und am 
    Ende einzuf\"ugen.  Die Klasse std::stack zeichnet sich daraus
    aus, das immer nur das oberste Elemente verändert werden kann (entfernen,
    hinzufügen usw.). Dies ist auch in std::deque möglich. Dies Voraussetzung
    hierzu ist die Beschränkung, das nur die Befehle für das obere Element
    verwendet werden. Dadurch ist es möglich std::stack durch std::deque zu
    implementieren. 

  \item
    \texttt{std::deque} ist meist durch eine Doppelt-Verlinkte Liste
    implementiert. Einf\"ugen von Elementen l\"auft deswegen immer
    in $\mathcal{O}(1)$. Siehe auch die entsprechende
    implementierung der ADT Liste. \texttt{std::vector} muss
    schnellen zuf\"alligen Zugriff auf die Elemente garantieren. Hierzu
    arbeitet im Hintergrund meist ein Array. Einf\"ugen von Elementen
    in ein Array kostet im Schlimmsten Fall $\mathcal{O}(n)$ Zeit.
    Falls der Vektor ausreichend speicher vorallokiert hat, so
    ist er in der Lage, Elemente schnell am Ende einzuf\"ugen. Auf diesem Weg 
    kann eine amortisierte Laufzeit von $\mathcal{O}(1)$ erm\"oglicht werden.


\end{enumerate}
\end{document}
