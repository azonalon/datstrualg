\documentclass[11pt]{article}
\usepackage[utf8]{inputenc}
\usepackage[english, ngerman]{babel}
\usepackage{amsmath,amsthm,verbatim,amssymb,amsfonts,amscd}
\usepackage{enumerate}
\usepackage{listings}
\usepackage{courier}
\lstset{
  numbers=left,
  language=C,
  basicstyle=\footnotesize\ttfamily,
  breaklines=true,
  morekeywords={function}
}
\newcommand{\abs}[1]{\left| #1 \right| }


\author{
  Felix Schrader, 3053850 \\ 
  Jens Duffert, 2843110 \\
  Eduard Sauter, 3053470
}
\title{Datenstrukturen und Algorithmen: Haus\"ubung 2}
\begin{document}
\maketitle

\subsection*{Aufgabe 1}

Hier kommt Aufgabe 1 hin!

\subsection*{Aufgabe 3}
%
Es sei $f(n)$ die Laufzeitfunktion des jeweiligen Algortihmus.
  \subsubsection*{\texttt{Linear-Search}}
    \begin{description}
      \item[Best-Case] Das gesuchte Element ist an erster Position. Die Laufzeit
        ist folglich $\Theta(1)$, da nur ein Element gepr\"uft wird.

      \item[Worst-Case] Das gesuchte Element ist an letzter Position. Es werden
        also $n$ Elemente \"uberpr\"uft. Dies bedeutet $\Theta(n)$.
        
      \item[Average-Case] 
        F\"ur einen beliebigen Index $i$ ist die Wahrscheinlichkeit, den 
        Wert $x$ an dieser Stelle zu finden nach Vorraussetzung $\frac{1}{n}$.
        Es sei $j$ der Index von $x$.  Also $P(j=i) = \frac{1}{n}$. 
        Damit ist der Erwartungswert 
        %
        \begin{align*}
          E(j)  = \sum_{i=0}^{n-1} (i + 1) P(j=i) = \frac{1}{n} \sum_{i=1}^{n} i 
                                                  = \frac{n + 1}{2} \\
        \end{align*}
        %
        Also ist die Laufzeitfunktion von \texttt{Linear-Search} in 
        $\Theta(n)$.

    \end{description}
  \subsubsection*{\texttt{Randomized-Search}}
    \begin{description}
      \item[Best-Case] Der erste Index ist nach dem ersten Versuch derjenige von
        $x$. Also ist in diesem fall wieder $f(n) \in \Theta(1)$. 

      \item[Worst-Case] Der Wert wird nie gefunden, weil zum Beispiel der
        Zuf\"allsgenerator in jedem Iterationsschritt 1 ausgibt und 
        $x$ nicht den ersten Index belegt.

      \item[Average-Case] 
        Es sei $q$ die Anzahl der ben\"otigten Schritte um $x$ zu finden. 
        In jedem Schritt ist die Wahrscheinlichkeit $x$ zu finden nach Vorraussetzung
        wieder $\frac{1}{n}$. F\"ur $P(q=i)$ muss noch die Wahrscheinlichkeit
        ber\"ucksichtigt werden, dass die Iteration nicht abgebrochen ist, weil $x$
        bereits gefunden wurde. Diese ist nach $i$ Schritten 
        $\left( 1 - \frac{1}{n} \right)^i$. 
        Da die ersten $i-1$ Zufallsexperimente unabh\"angig von den $i$-ten
        Versuch sind, ist 
        %
        \begin{align*}
          P(q=i)= \frac{1}{n} \left( 1 - \frac{1}{n} \right)^( i - 1 )
        \end{align*}
        %
        Damit ergibt sich der Erwartungswert zu
        %
        \begin{align*}
          E(q) & = \sum_{i=1}^{\infty} i P(q=i)  \\
               & = \left( 1 - \frac{1}{n} \right)^{-1} 
          \sum_{i=1}^{\infty} \frac{i}{n} \left( 1- \frac{1}{n} \right)^i  \\
          & = \frac{1}{ 1-\frac{1}{n} } 
          \frac{1-\frac{1}{n} }{\left( \frac{1}{n} \right)^2} = n^2 
        \end{align*}
        %
        Damit ist $f(n) \in \Theta(n^2)$
    \end{description}

\end{document}
