\documentclass[11pt]{article}
\usepackage[utf8]{inputenc}
\usepackage[english, ngerman]{babel}
\usepackage{amsmath,amsthm,verbatim,amssymb,amsfonts,amscd}
\usepackage{enumerate}
\usepackage{listings}
\usepackage{courier}
\usepackage[margin=1in]{geometry}
\lstset{
  numbers=left,
  language=C,
  basicstyle=\footnotesize\ttfamily,
  breaklines=true,
  morekeywords={function}
}
\newcommand{\abs}[1]{\left| #1 \right| }
\setlength{\parindent}{0pt} 

\author{
  Felix Schrader, 3053850 \\ 
  Jens Duffert, 2843110 \\
  Eduard Sauter, 3053470
}
\title{Datenstrukturen und Algorithmen: Haus\"ubung 2}
\begin{document}
\maketitle

\subsection*{Aufgabe 1}

\begin{enumerate}[(a)]
  \item Es gilt:
    \begin{align*}
      2^{n+d} = 2^d \cdot 2^n
    \end{align*}
    Wählt man $c = 42 \cdot 2^d$ und $n_0 = 1$, so ist
    \begin{align*}
      f(n)=2^d \cdot 2^n \leq c \cdot g(n) = 42 \cdot 2^d \cdot 2^n
    \end{align*}
    für alle $n \geq n_0$, da man auf beiden Seiten durch $2^d \cdot 2^n$ teilen
    kann und die wahre Aussage $1 \leq 42$ erhält.
  \item Es gilt:
    \begin{align*}
      \log_a (n) = \frac{1}{\log_b (a)} \cdot \log_b (n)
    \end{align*}
    In der Stundenübung 2.1 haben wir gezeigt:
    \begin{align*}
      \mathcal{O}(f(n)) = \mathcal{O}(c \cdot f(n)), \quad c > 0
    \end{align*}
    Wählt man also $f(n) = \log_a (n)$ und $c = 1/\log_b (a) > 0$, so
    erhält man:
    \begin{align*}
      \mathcal{O}(\log_a (n)) = \mathcal{O}(\log_b (n))
    \end{align*}
  \item In der Vorlesung wurde als Rechenregel für Landau-Symbole eingeführt:
    \begin{align*}
      \Theta (f(n)) + \Theta (g(n)) = \Theta (f(n)+g(n))
    \end{align*}
    Indem man diese Regel iterativ anwendet, erhält man für die Summe:
    \begin{align*}
      \sum_{i=1}^n \Theta (i) &= \Theta \left(\sum_{i=1}^n i\right)
      \\ &= \Theta \left(\frac{n(n+1)}{2}\right)
      \\ &= \Theta \left(\frac{n^2}{2} + \frac{n}{2}\right)
    \end{align*}
    Wählt man $c = 1$ und $n_0 = 1$, so erkennt man:
    \begin{align*}
      &\frac{n^2}{2} + \frac{n}{2} \in \mathcal{O}(n^2),\quad\text{da}
      \\ &\frac{n^2}{2} + \frac{n}{2} \leq n^2 \quad\forall n \geq n_0
    \end{align*}
    Mit $c = 1/2$ und $n_0 = 1$ sieht man außerdem:
    \begin{align*}
      &\frac{n^2}{2} + \frac{n}{2} \in \Omega (n^2),\quad\text{da}
      \\ &\frac{n^2}{2} + \frac{n}{2} \geq\frac{n^2}{2}\quad\forall n\geq n_0
      \\ &\Leftrightarrow \frac{n}{2} \geq 0 \quad\forall n \geq n_0
    \end{align*}
    Also ist insgesamt:
    \begin{align*}
      \sum_{i=1}^n \Theta (i) &= \Theta \left(\frac{n^2}{2}+\frac{n}{2}\right)
      \\ &= \Theta (n^2)
    \end{align*}
  \item Es gilt:
    \begin{align*}
      \sum_{r=0}^{n-1} \sum_{s=r}^{n-1} \sum_{i=r}^{s} 1 &= \sum_{r=0}^{n-1}
        \sum_{s=r}^{n-1} (s-r+1)
      \\ &= \sum_{r=0}^{n-1} \sum_{s=0}^{n-r-1} (1+s)
      \\ &= \sum_{r=0}^{n-1} (n-r) + \frac{(n-r-1)(n-r)}{2}
      \\ &= \sum_{r=0}^{n-1} \left(\frac{n^2+n}{2}-\frac{2n+1}{2}r+
        \frac{r^2}{2}\right)
      \\ &= \frac{n(n^2+n)}{2}-\frac{n(n-1)(2n+1)}{4}+\frac{n(n-1)(2n-1)}{12}
      \\ &= \frac{n^3}{6} + \frac{n^2}{2} + \frac{n}{3}
    \end{align*}
     Wählt man $c = 1/6$ und $n_0 = 1$, so erkennt man:
     \begin{align*}
       &\sum_{r=0}^{n-1} \sum_{s=r}^{n-1} \sum_{i=r}^{s} 1 \in \Omega (n^3)
       \\ &\Leftrightarrow \frac{n^3}{6} + \frac{n^2}{2} + \frac{n}{3} \geq
         \frac{n^3}{6} \quad\forall n \geq n_0
       \\&\Leftrightarrow\frac{n^2}{2}+\frac{n}{3}\geq 0\quad\forall n\geq n_0
     \end{align*}
\end{enumerate}

\subsection*{Aufgabe 2}
%

Gegeben ist:\\
\begin{align*}
f(n) \in  \mathcal{O}(g(n)) &\Leftrightarrow \lim\limits_{n \rightarrow \infty}{\text{sup}\, \dfrac{f(n)}{g(n)} < \infty}\\[0.5cm]
f(n) \in  \Omega(g(n)) &\Leftrightarrow \lim\limits_{n \rightarrow \infty}{\text{inf}\, \dfrac{f(n)}{g(n)} >\, 0}
\end{align*}
\begin{enumerate}[a)]
%
\item Die Funktion die gegeben ist sieht wie folgt aus:
\begin{align*}
e ^{n} \notin \mathcal{O}(n^{c})  \quad \forall c \in \mathrm{N}
\end{align*}
Es wird nun der Limes betrachtet.
\begin{align*}
\lim\limits_{n \rightarrow \infty}{\dfrac{e^{n}}{n^{c}}} \Rightarrow \infty
\end{align*}
Da der Grenzwert $\infty$ beträgt, wird L`Hospital verwendet.
\begin{align*}
\lim\limits_{n \rightarrow \infty}{\dfrac{e^{n}}{n^{c}}} &= \lim\limits_{n \rightarrow \infty}{\dfrac{e^{n}}{c \cdot n^{c - 1}}} = \lim\limits_{n \rightarrow \infty}{\dfrac{e^{n}}{c \cdot (c-1) \cdot n^{c - 2}}} = \ldots \\
& =\lim\limits_{n \rightarrow \infty}{\dfrac{e^{n}}{c \cdot (c-1) \ldots \cdot n^{c - c}}} = \lim\limits_{n \rightarrow \infty}{\dfrac{e^{n}}{c!}}
\end{align*}
Da der Limes $\infty$ ist, folgt daraus das die Funktion $e^{n} \notin \mathcal{O}$ ist (nach der oben genannten Definition.)
\item  Die Funktion die gegeben ist sieht wie folgt aus:
\begin{align*}
\text{log}(n) \notin \Omega(n)
\end{align*}
Wie im Hinweis geschrieben, wird der normale Limes betrachtet. 
\begin{align*}
\lim\limits_{n \rightarrow \infty}{\dfrac{\text{log}(n)}{n}} \Rightarrow \infty
\end{align*}
Nun wird L'Hospital angewendet.
\begin{align*}
\lim\limits_{n \rightarrow \infty}{\dfrac{\text{log}(n)}{n}} = \lim\limits_{n \rightarrow \infty}{\dfrac{\dfrac{1}{n}}{1}} = \lim\limits_{n \rightarrow \infty}{\dfrac{1}{n}}
\end{align*}
Es wird der Limes nun berechnet:
\begin{align*}
\lim\limits_{n \rightarrow \infty}{\dfrac{1}{n}} = 0
\end{align*}

Da der Limes 0 ist folgt daraus, dass die Funktion $\text{log}(n) \notin \Omega(n)$ ist (nach der oben genannten Definition).

\end{enumerate}
\subsection*{Aufgabe 3}
%
Es sei $f(n)$ die Laufzeitfunktion des jeweiligen Algorithmus.
  \subsubsection*{\texttt{Linear-Search}}
    \begin{description}
      \item[Best-Case] Das gesuchte Element ist an erster Position. Die Laufzeit
        ist folglich $\Theta(1)$, da nur ein Element gepr\"uft wird.

      \item[Worst-Case] Das gesuchte Element ist an letzter Position. Es werden
        also $n$ Elemente \"uberpr\"uft. Dies bedeutet $\Theta(n)$.
        
      \item[Average-Case] 
        F\"ur einen beliebigen Index $i$ ist die Wahrscheinlichkeit, den 
        Wert $x$ an dieser Stelle zu finden nach Vorraussetzung $\frac{1}{n}$.
        Es sei $j$ der Index von $x$.  Also $P(j=i) = \frac{1}{n}$. 
        Damit ist der Erwartungswert 
        %
        \begin{align*}
          E(j)  = \sum_{i=0}^{n-1} (i + 1) P(j=i) = \frac{1}{n} \sum_{i=1}^{n} i 
                                                  = \frac{n + 1}{2} \\
        \end{align*}
        %
        Also ist die Laufzeitfunktion von \texttt{Linear-Search} in 
        $\Theta(n)$.

    \end{description}
  \subsubsection*{\texttt{Randomized-Search}}
    \begin{description}
      \item[Best-Case] Der erste Index ist nach dem ersten Versuch derjenige von
        $x$. Also ist in diesem fall wieder $f(n) \in \Theta(1)$. 

      \item[Worst-Case] Der Wert wird nie gefunden, weil zum Beispiel der
        Zuf\"allsgenerator in jedem Iterationsschritt 1 ausgibt und 
        $x$ nicht den ersten Index belegt.

      \item[Average-Case] 
        Es sei $q$ die Anzahl der ben\"otigten Schritte um $x$ zu finden. 
        In jedem Schritt ist die Wahrscheinlichkeit $x$ zu finden nach Vorraussetzung
        wieder $\frac{1}{n}$. F\"ur $P(q=i)$ muss noch die Wahrscheinlichkeit
        ber\"ucksichtigt werden, dass die Iteration nicht abgebrochen ist, weil $x$
        bereits gefunden wurde. Diese ist nach $i$ Schritten 
        $\left( 1 - \frac{1}{n} \right)^i$. 
        Da die ersten $i-1$ Zufallsexperimente unabh\"angig von den $i$-ten
        Versuch sind, ist 
        %
        \begin{align*}
          P(q=i)= \frac{1}{n} \left( 1 - \frac{1}{n} \right)^{ i - 1 }
        \end{align*}
        %
        Damit ergibt sich der Erwartungswert zu
        %
        \begin{align*}
          E(q) & = \sum_{i=1}^{\infty} i P(q=i)  \\
               & = \left( 1 - \frac{1}{n} \right)^{-1} 
          \sum_{i=1}^{\infty} \frac{i}{n} \left( 1- \frac{1}{n} \right)^i  \\
          & = \frac{1}{ 1-\frac{1}{n} } 
          \frac{1-\frac{1}{n} }{\left( \frac{1}{n} \right)^2} = n^2 
        \end{align*}
        %
        Damit ist $f(n) \in \Theta(n^2)$
    \end{description}

\end{document}
