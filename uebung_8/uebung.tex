\documentclass[11pt]{article}
\usepackage[utf8]{inputenc}
\usepackage[english, ngerman]{babel}
\usepackage{amsmath,amsthm,verbatim,amssymb,amsfonts,amscd}
\usepackage{enumerate}
\usepackage{listings}
\usepackage{courier}
\usepackage[]{graphicx}
\usepackage[]{epstopdf}
\usepackage[margin=1in]{geometry}
\lstset{
  numbers=left,
  language=C,
  basicstyle=\footnotesize\ttfamily,
  breaklines=true,
  morekeywords={function, NIL, new, class, implements, var, true, false}
}
\newcommand{\abs}[1]{\left| #1 \right| }
\setlength{\parindent}{0pt} 

\author{
  Felix Schrader, 3053850 \\ 
  Jens Duffert, 2843110 \\
  Eduard Sauter, 3053470
}
\title{Datenstrukturen und Algorithmen: Haus\"ubung 8}
\begin{document}
\maketitle
\subsection*{Aufgabe 1}
In der Stundenübung 8.1(b) haben wir für die mittlere Zugriffszahl bei der
erfolgreichen Suche gezeigt:
\begin{align*}
  m(x \vert x \in S) = \frac{\sum_{x \in S} p(x) z(x)}{\sum_{x \in S} p(x)}
\end{align*}
Dabei ist $x$ ein Schlüssel, $S$ die Menge aller Schlüssel, $p(x)$ die
Wahrscheinlichkeit der Suche nach $x$ und $z(x)$ die Anzahl der benötigten
Zugriffe um $x$ zu finden. Für die lineare Suche entspricht $z(x)$ immer der
Position von $x$ in der Tabelle (beginnend bei 1). In diesem Beispiel ergibt
sich:
\begin{align*}
  \sum_{x \in S} p(x) &= 1 \\
  \Rightarrow m(x \vert x \in S) &= \sum_{x \in S} p(x) z(x)
\end{align*}
\begin{enumerate}[a)]
  \item 
    In der gegebenen Reihenfolge hat man die Werte:
    \begin{table}[h!]
      \centering
      \begin{tabular}{|c|c|c|c|c|c|c|c|c|c|c|c|c|}
        \hline 
        $x$ & J & Y & O & S & E & K & N & T & R & I & C & B \\ 
        \hline 
        $p(x)$ & 0,048 & 0,049 & 0,2 & 0,32 & 0,08 & 0,008 & 0,068 & 0,069 &
        0,07 & 0,028 & 0,051 & 0,009 \\ 
        \hline 
        $z(x)$ & 1 & 2 & 3 & 4 & 5 & 6 & 7 & 8 & 9 & 10 & 11 & 12 \\ 
        \hline 
      \end{tabular} 
    \end{table}

    Damit ergibt sich:
    \begin{align*}
      m(x \vert x \in S) = 5,081
    \end{align*}
  \item 
    Ordnet man die Schlüssel nach steigender Wahrscheinlichkeit, so sind die
    Zugriffszahlen und Wahrscheinlichkeiten:
    \begin{table}[h!]
      \centering
      \begin{tabular}{|c|c|c|c|c|c|c|c|c|c|c|c|c|}
        \hline 
        $x$ & K & B & I & J & Y & C & N & T & R & E & O & S \\ 
        \hline 
        $p(x)$ & 0,008 & 0,009 & 0,028 & 0,048 & 0,049 & 0,051 & 0,068 & 0,069 &
        0,07 & 0,08 & 0,2 & 0,32 \\ 
        \hline 
        $z(x)$ & 1 & 2 & 3 & 4 & 5 & 6 & 7 & 8 & 9 & 10 & 11 & 12 \\ 
        \hline 
      \end{tabular} 
    \end{table}

    Als Ergebnis erhält man:
    \begin{align*}
      m(x \vert x \in S) = 9,351
    \end{align*}
    \newpage
  \item 
    Für den nach fallender Wahrscheinlichkeit sortierten Schlüssel sind die
    Werte:
    \begin{table}[h!]
      \centering
      \begin{tabular}{|c|c|c|c|c|c|c|c|c|c|c|c|c|}
        \hline 
        $x$ & S & O & E & R & T & N & C & Y & J & I & B & K \\ 
        \hline 
        $p(x)$ & 0,32 & 0,2 & 0,08 & 0,07 & 0,069 & 0,068 & 0,051 & 0,049 &
        0,048 & 0,028 & 0,009 & 0,008 \\ 
        \hline 
        $z(x)$ & 1 & 2 & 3 & 4 & 5 & 6 & 7 & 8 & 9 & 10 & 11 & 12 \\ 
        \hline 
      \end{tabular} 
    \end{table}

    Die mittlere Zugriffszahl der erfolgreichen Suche ist dann:
    \begin{align*}
      m(x \vert x \in S) = 3,649
    \end{align*}
\end{enumerate}
\subsection*{Aufgabe 2}
\begin{enumerate}[a)]
  \item  $ $
    \begin{enumerate}[i)]
      \item Die mittlere Zugriffszahl aller Suchanfragen ergibt sich analog
        zur Berechnung ohne Marker wie in Aufgabe 1. Nach Befragung seines
        Lieblings Computeralgebrasystems erh\"alt man
        %
        \begin{align*}
          m_\text{alle} = \frac{741}{100}
        \end{align*}
        %
      \item Hier muss man alle Elemente mit Marker 0 aus der Rechnung 
        herausnehmen und das ganze normieren. 
        Es sei $\mu_i$ der $i$-te Marker. Die Formel ist dann
        %
        \begin{align*}
          m_\text{erfolgreich} = \frac{\sum_{i=0}^{11} \mu_i p_i (i+1)}{
          \sum_{i=0}^{11} \mu_i p_i} = \frac{373}{64}
        \end{align*}
        %
      \item F\"ur diese Rechnung m\"ussen genau alle Marker mit Wert 0
        ber\"ucksichtig werden, und alle Elemente die nicht im Array sind.
        Dazu sei dann $\lambda = 1 - \sum_{i=0}^{11} p_i$.
        Man erh\"alt dann wieder mit Normierung:
        %
        \begin{align*}
          m_\text{erfolglos} = \frac{12 \lambda + \sum_{i=0}^{11} (1-\mu_i) p_i (i+1) }{
          \lambda + \sum_{i=0}^{11} (1-\mu_i) p_i } = \frac{21}{2}
        \end{align*}
        %
    \end{enumerate} 
\newpage
  \item $ $
    \begin{table}[h!]
      \centering
      \begin{tabular}{r | c c c c c c c c c c c c c}
        \hline\hline
        \textbf{Index} & 0 & 1 & 2 & 3 & 4 & 5 & 6 & 7 & 8 & 9 & 10 & 11 \\
        \hline
        \textbf{Sequenz} & G & F & C & A & D & E & L & R & K & N & O & M \\
        \textbf{Marker} & 1 & 1 & 1 & 0 & 1 & 1 & 0 & 0 & 1 & 1 & 0 & 1 \\
        \hline
        Gesucht: D  & G & F & C & D & A & E & L & R & K & N & O & M \\
                    & 1 & 1 & 1 & 1 & 0 & 1 & 0 & 0 & 1 & 1 & 0 & 1 \\
        \hline
         A  & G & F & C & A & D & E & L & R & K & N & O & M \\
                    & 1 & 1 & 1 & 0 & 1 & 1 & 0 & 0 & 1 & 1 & 0 & 1 \\
        \hline
         S  & G & F & C & A & D & E & L & R & K & N & S & M \\
                    & 1 & 1 & 1 & 0 & 1 & 1 & 0 & 0 & 1 & 1 & 0 & 1 \\
        \hline
         W  & G & F & C & A & D & E & L & R & K & N & W & M \\
                    & 1 & 1 & 1 & 0 & 1 & 1 & 0 & 0 & 1 & 1 & 0 & 1 \\
        \hline
         E  & G & F & C & A & E & D & L & R & K & N & W & M \\
                    & 1 & 1 & 1 & 0 & 1 & 1 & 0 & 0 & 1 & 1 & 0 & 1 \\
        \hline
         L  & G & F & C & A & E & L & D & R & K & N & W & M \\
                    & 1 & 1 & 1 & 0 & 1 & 0 & 1 & 0 & 1 & 1 & 0 & 1 \\
        \hline
         F  & F & G & C & A & E & L & D & R & K & N & W & M \\
                    & 1 & 1 & 1 & 0 & 1 & 0 & 1 & 0 & 1 & 1 & 0 & 1 \\
        \hline
         E  & F & G & C & E & A & L & D & R & K & N & W & M \\
                    & 1 & 1 & 1 & 1 & 0 & 0 & 1 & 0 & 1 & 1 & 0 & 1 \\
        \hline
         N  & F & G & C & E & A & L & D & R & N & K & W & M \\
                    & 1 & 1 & 1 & 1 & 0 & 0 & 1 & 0 & 1 & 1 & 0 & 1 \\
        \hline
         L  & F & G & C & E & L & A & D & R & N & K & W & M \\
                    & 1 & 1 & 1 & 1 & 0 & 0 & 1 & 0 & 1 & 1 & 0 & 1 \\
        \hline
         A  & F & G & C & E & A & L & D & R & N & K & W & M \\
                    & 1 & 1 & 1 & 1 & 0 & 0 & 1 & 0 & 1 & 1 & 0 & 1 \\
        \hline
         B  & F & G & C & E & A & L & D & R & N & K & B & M \\
                    & 1 & 1 & 1 & 1 & 0 & 0 & 1 & 0 & 1 & 1 & 0 & 1 \\
        \hline
      \end{tabular}
    \end{table}
    
\newpage
  \item $ $
    \lstinputlisting{search_sequence.js}
\end{enumerate} 

\newpage

\subsection*{Aufgabe 3}
In dieser Aufgabe soll eine Tabelle der Länge 12 erstellt und mittels Hashing
die Werte eingefügt werden. Dies wird anhand eines Wertes gezeigt, da die 
anderen Werte äquivalent berechnet werden.

\begin{align*}
	h(z) &= (2z+5) \mod 11\\
	z &= 12\\
	h(12) &= (2 \cdot 12 + 5) \mod 11\\
	h(12) &= 7 
\end{align*}
Dies wurde auch mit den anderen Werten gemacht. Diese werden nun in die Tabelle eingeschrieben.
\begin{table}[h!]
	\centering
	\begin{tabular}{|c|c|c|c|c|c|c|c|c|c|c|c|c|c|c|}
		\hline\hline
		\textbf{Eintrag}&\textbf{h(s)} & 0 & 1 & 2 & 3 & 4 & 5 & 6 & 7 & 8 & 9 & 10 & 11 \\
		\hline
		12 & 7 & & & & & & & & 12 & & & &\\
		\hline
		44 & 5 & & & & & & 44 & & & & & &\\
		\hline
		13 & 9 & & & & & & & & & & 13 & &\\
		\hline
		88 & 5 & & & & & & $\rightarrow$ & 88 & & & & &\\
		\hline
		23 & 7 & & & & & & & & $\rightarrow$ & 23 & & &\\
		\hline
		94 & 6 & & & & & & & $\rightarrow$ & $\rightarrow$ & $\rightarrow$ & $\rightarrow$ & 94 &\\
		\hline
		11 & 5 & & & & & & $\rightarrow$ & $\rightarrow$ & $\rightarrow$ & $\rightarrow$ & $\rightarrow$ & $\rightarrow$ & 5 \\
		\hline
		39 & 6 & 39 & & & & & & $\rightarrow$ & $\rightarrow$ & $\rightarrow$ & $\rightarrow$ & $\rightarrow$ & $\rightarrow$ \\
		\hline
		20 & 1 & & 20 & & & & & & & & & &\\
		\hline
		16 & 4 & & & & & 16 & &  & & & & &\\
		\hline
		9 & 1 & & $\rightarrow$ & 1 & & & & & & & & &\\
		\hline
	\end{tabular}
\end{table}

\end{document}
