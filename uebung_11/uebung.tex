\documentclass[11pt]{article}
\usepackage[utf8]{inputenc}
\usepackage[english, ngerman]{babel}
\usepackage{amsmath,amsthm,verbatim,amssymb,amsfonts,amscd}
\usepackage{enumerate}
\usepackage{listings}
\usepackage{courier}
\usepackage{graphicx}
\usepackage{epstopdf}
\usepackage[margin=1in]{geometry}
\lstset{
  numbers=left,
  language=C,
  basicstyle=\footnotesize\ttfamily,
  breaklines=true,
  morekeywords={function, NIL}
}
\newcommand{\abs}[1]{\left| #1 \right| }
\setlength{\parindent}{0pt} 

\author{
  Felix Schrader, 3053850 \\ 
  Jens Duffert, 2843110 \\
  Eduard Sauter, 3053470
}
\title{Datenstrukturen und Algorithmen: Haus\"ubung 11}
\begin{document}
\maketitle
\subsection*{Aufgabe 3}
  \begin{enumerate}[a)]
    \item 
      \begin{align*}
        \texttt{i=0:} & & \\
        \texttt{a} & & \\ 
        \texttt{b} & & \\ 
        \vdots & & \\ 
        \texttt{g} & \rightarrow \texttt{graph} & \\ 
        \vdots & & \\ 
        \texttt{p} & \rightarrow \texttt{position, partition, pivot, parität}
          & \rightarrow \texttt{i=1 (p)} \\ 
        \vdots & & \\ 
        \texttt{s} & \rightarrow \texttt{stack, speicher, stapelspeicher}
          & \rightarrow \texttt{i=1 (s)} \\ 
        \vdots & & \\ 
        \texttt{z} & & \\ 
        \texttt{i=1 (p):} & & \\
        \texttt{a} & \rightarrow \texttt{partition, parität} & \rightarrow
          \texttt{i=2 (pa)} \\ 
        \vdots & & \\ 
        \texttt{i} & \rightarrow \texttt{pivot} & \\ 
        \vdots & & \\ 
        \texttt{o} & \rightarrow \texttt{position} & \\ 
        \vdots & & \\ 
        \texttt{i=2 (pa):} & & \\
        \vdots & & \\ 
        \texttt{r} & \rightarrow \texttt{partition, parität} & \rightarrow
          \texttt{i=3 (par)} \\ 
        \vdots & & \\ 
      \end{align*}
      \begin{align*}
        \texttt{i=3 (par):} & & \\
        \vdots & & \\ 
        \texttt{i} & \rightarrow \texttt{parität} & \\ 
        \vdots & & \\ 
        \texttt{t} & \rightarrow \texttt{partition} & \\ 
        \vdots & & \\ 
        \texttt{i=1 (s):} & & \\ 
        \vdots & & \\ 
        \texttt{p} & \rightarrow \texttt{speicher} & \\ 
        \vdots & & \\ 
        \texttt{t} & \rightarrow \texttt{stack, stapelspeicher} & \rightarrow
          \texttt{i=2 (st)} \\ 
        \vdots & & \\ 
        \texttt{i=2 (st):} & & \\ 
        \texttt{a} & \rightarrow \texttt{stack, stapelspeicher} & \rightarrow
          \texttt{i=3 (sta)} \\ 
        \vdots & & \\ 
        \texttt{i=3 (sta):} & & \\
        \vdots & & \\ 
        \texttt{c} & \rightarrow \texttt{stack} & \\ 
        \vdots & & \\ 
        \texttt{p} & \rightarrow \texttt{stapelspeicher} & \\ 
        \vdots & & \\ 
      \end{align*}
      \texttt{Sortierung: graph, parität, partition, pivot, position, speicher,
        stack, stapelspeicher}
    \item 
      Der Algorithmus lässt sich auf Zahlen übertragen, indem man statt der
      Sortierung \texttt{a,\dots z} die Reihenfolge \texttt{0,\dots 9}
      verwendet. Allerdings muss man vorher die Zahl mit den meisten Stellen
      finden und bei kürzeren Zahlen so viele Nullen vorne hinzufügen, dass sie
      gleich lang sind (sonst wäre z.B. \texttt{9>10} möglich).
    \item 
  \end{enumerate}
\end{document}
